\documentclass[10pt]{article}
\begin{document}


El proyecto requirió realizar un sistema informático para la administración de una papelería, de manera que los datos cambien en tiempo real y se mantengan consistentes siguiendo ciertas reglas de negocio. Para hacer lo anterior posible se siguieron las siguientes condiciones.

\begin{itemize}
\item El sistema maneja el inventariado de los productos almacenando sus características   	principales y su proveedor, a la vez que mantiene sus existencias(stock).
\item Tiene un registro de los proveedores y su información.
\item Almacena cierta información de los clientes.
\item Hace posible un proceso de ventas, de manera que en la venta se tenga el cliente, los 	  productos que la conforman y un identificador único. Los productos vendidos se deben retirar del inventario. Además se deben calcular el monto total y el monto por cada artículo. Lo anterior se genera en una vista. 
\end{itemize}

Para resolver los puntos anteriores se implementó lo siguiente:

 \begin{itemize}
\item La base de datos se almacenó en \textit{postgreSQL} manejándola con \textit{pgAdmin}.
\item La interfaz de usuario se realizó con \textit{Java}.
\item Los diagramas se trazaron en \textit{Día}.
\item Para el entorno web se usó... 
\item La documentación se redactó en \LaTeX{}.
\end{itemize}

\end{document}